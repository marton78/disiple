\documentclass{article}
\usepackage[utf8]{inputenc}
\usepackage[english]{babel}
\usepackage[a4paper, total={5in, 10in}]{geometry}

\usepackage{amsmath}
\usepackage{amsthm}

\theoremstyle{definition}
\newtheorem*{definition}{Definition}
\newtheorem*{theorem}{Theorem}
\newtheorem*{example}{Example}

\setlength{\parindent}{0em}
\pagenumbering{gobble}

\title{FIR Filtering with Polynomial Coefficients}
\author{Márton Danóczy}
\date{January 2014}

\begin{document}

\maketitle

\begin{definition}
A filter shall be called \emph{polynomial filter} if all of its coefficients can be expressed as a polynomial.

\[ w_i = \sum_p a_p i^p \]

An example of such a filter would be the Welch window function, a quadratic polynomial which reaches unity in the middle of the window and zero at the samples just outside the span of the window.

\end{definition}

\begin{theorem}
Convolution with polynomial filters can be computed recursively.
\end{theorem}

\begin{proof}
We begin by defining $z_{p,t}$ as the convolution of the vector $(1, 2^p, \dots, n^p)^\top$ with the signal $x$:

\begin{align*}
(w*x)[t] &= \sum_{i=1}^n w_i x_{t-(i-1)}
    = \sum_{i=1}^n \sum_p a_p i^p x_{t-(i-1)} \\
    &= \sum_p a_p \sum_{i=1}^n i^p x_{t-(i-1)}
    = \sum_p a_p z_{p,t}
\end{align*}

Now, by shifting indices we arrive at the recurrence relation:

\begin{align*}
z_{p,t} &\equiv \sum_{i=1}^n i^p x_{t-(i-1)} = \sum_{i=0}^{n-1} (i+1)^p x_{t-i} = \sum_{i=0}^{n-1} \sum_{k=0}^{p} \binom{p}{k} i^k x_{t-i} \\
 &= \sum_{k=0}^{p} \binom{p}{k} \left[ \sum_{i=0}^{n-1} i^k x_{t-i} \right]
 = \sum_{k=0}^{p} \binom{p}{k} \left[ \sum_{i=1}^{n} i^k x_{t-i} - n^k x_{t-n} + 0^k x_t\right] \\
 &= \sum_{k=0}^{p} \binom{p}{k} \left[ z_{k,t-1} - n^k x_{t-n} + 0^k x_t\right] \\
 &= z_{p,t-1} - n^p x_{t-n} + \sum_{k=0}^{p-1} \binom{p-1}{k} \left[ z_{k,t-1} - n^k x_{t-n} + 0^k x_t\right] \\
 &= z_{p,t-1} - n^p x_{t-n} + z_{p-1,t} \quad\mathrm{if}\quad p \geq 1
\end{align*}

\end{proof}

\begin{example}
For a quadratic polynomial filter, the recurrence is computed as follows:

\begin{align*}
z_{0,t} &= z_{0,t-1} - x_{t-n} + x_t \\
z_{1,t} &= z_{1,t-1} - n x_{t-n} + z_{0,t} \\
z_{2,t} &= z_{2,t-1} - n^2 x_{t-n} + z_{1,t}
\end{align*}

Note that due to the lack of divisions this recurrence can be used to implement polynomial filtering in the domain of integers.

\end{example}

\end{document}
